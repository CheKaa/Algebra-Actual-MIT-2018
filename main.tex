\documentclass[10pt,a4paper,oneside]{book}
\usepackage[a4paper,includeheadfoot,top=10mm,bottom=10mm,left=10mm,right=10mm]{geometry}
\usepackage[utf8]{inputenc}
\usepackage[russian]{babel}
\usepackage{amsmath,amsthm,amssymb,amscd,array}
\usepackage{latexsym}
\usepackage{multicol} % нумерция в нескольких колонках
\usepackage{graphicx} 
%\usepackage{pdfsync}
\usepackage{pgf}
\usepackage{tikz}
\usepackage{tikz-cd}
\usetikzlibrary{arrows,backgrounds,patterns,matrix,shapes,fit,calc,shadows,plotmarks}
\usepackage{hyperref} % гиперссылки
\usepackage{cmap}       % Поддержка поиска русских слов в PDF (pdflatex)
\usepackage{indentfirst}% Красная строка в первом абзаце
\usepackage{misccorr}
\usepackage{arydshln} % штрихованые линии в массивах
\usepackage{mathtools} % выравнивание в матрицах
\usepackage{ccaption}
\usepackage{fancyhdr}
\usepackage{comment}
\usepackage{xcolor}
\hypersetup{
    colorlinks,
    linkcolor={blue!50!black},
    citecolor={blue!50!black},
    urlcolor={red!80!black}
}
% цвета для ссылок

\newtheorem{upr}{Упражнение}
\newtheorem{predl}{Предложение}
\newtheorem{komment}{Комментарий}
\newtheorem{conj}{Гипотеза}
\newtheorem{notation}{Обозначение}


\theoremstyle{definition}
\newtheorem{kit}{Кит}
\newtheorem*{rem}{Замечание}
\newtheorem{zad}{Задача}
\newtheorem*{defn}{Определение}
\newtheorem*{fact}{Факт}
\newtheorem{thm}{Теорема}
\newtheorem*{thmm}{Теорема}
\newtheorem{lem}{Лемма}
\newtheorem{cor}{Следствие}



\renewcommand{\proofname}{Доказательство}
\renewcommand{\mod}{\,\operatorname{mod}\,}
\renewcommand{\Re}{\operatorname{Re}}
\newcommand{\mf}[1]{\mathfrak{#1}}
\newcommand{\mcal}[1]{\mathcal{#1}}
\newcommand{\mb}[1]{\mathbb{#1}}
\newcommand{\mc}[1]{\mathcal{#1}}
\newcommand{\tbf}[1]{\textbf{#1}}
\newcommand{\ovl}{\overline}
\newcommand{\Spec}{\operatorname{Spec}}
\newcommand{\K}{\operatorname{K_0}}
\newcommand{\witt}{\operatorname{W}}
\newcommand{\gw}{\operatorname{GW}}
\newcommand{\coh}{\operatorname{H}}
\newcommand{\dist}{\operatorname{dist}}
\newcommand{\cl}{\operatorname{Cl}}
\newcommand{\Vol}{\operatorname{Vol}}
\newcommand\tgg{\mathop{\rm tg}\nolimits}
\newcommand\ccup{\mathop{\cup}}
\newcommand{\id}{\operatorname{Id}}
\newcommand{\lcm}{\operatorname{lcm}}
\newcommand{\chr}{\operatorname{char}}
\newcommand{\rk}{\operatorname{rank}}
\DeclareMathOperator{\Coker}{Coker}
\DeclareMathOperator{\Ker}{Ker}
\newcommand{\im}{\operatorname{Im}}
\renewcommand{\Im}{\operatorname{Im}}
\newcommand{\Tr}{\operatorname{Tr}}
\newcommand{\re}{\operatorname{Re}}
\newcommand{\tr}{\operatorname{Tr}}
\newcommand{\ord}{\operatorname{ord}}
\newcommand{\Stab}{\operatorname{Stab}}
\newcommand{\orb}{\operatorname{\mathcal O}}
\newcommand{\Fix}{\operatorname{Fix}}
\newcommand{\Hom}{\operatorname{Hom}}
\newcommand{\End}{\operatorname{End}}
\newcommand{\Aut}{\operatorname{Aut}}
\newcommand{\Inn}{\operatorname{Inn}}
\newcommand{\Out}{\operatorname{Out}}
\newcommand{\GL}{\operatorname{GL}}
\newcommand{\SL}{\operatorname{SL}}
\newcommand{\SO}{\operatorname{SO}}
\newcommand{\Sym}{\operatorname{Sym}}
\newcommand{\Adj}{\operatorname{Adj}}
\newcommand{\Disc}{\operatorname{Disc}}
\newcommand{\cnt}{\operatorname{cont}}

\newcommand{\di}{\mathop{\,\scalebox{0.85}{\raisebox{-1.2pt}[0.5\height]{\vdots}}\,}}

\newcommand{\ndi}{\mathop{\not\scalebox{0.85}{\raisebox{-1.2pt}[0.5\height]{\vdots}}\,}}
\newcommand{\nequiv}{\not \equiv}
\newcommand{\Nod}{\operatorname{\text{НОД}}}
\newcommand{\Nok}{\operatorname{\text{НОК}}}
\newcommand{\sgn}{\operatorname{sgn}}


\def\llq{\textquotedblleft} 
\def\rrq{\textquotedblright} 
\def\exm{\noindent {\bf Примеры:}}


\def\Cb{\ovl{C}}
\def\ffi{\varphi}
\def\pa{\partial}
\def\V{\bf V}
\def\La{\Lambda}
\def\eps{\varepsilon}
\def\del{\delta}
\def\Del{\Delta}
\def\A{\EuScript{A}}
\def\lan{\left\langle }
\def\ran{\right\rangle}
\def\bar{\begin{array}}
\def\ear{\end{array}}
\def\beq{\begin{equation}}
\def\eeq{\end{equation}}
\def\thrm{\begin{thm}}
\def\ethrm{\end{thm}}
\def\dfn{\begin{defn}}
\def\edfn{\end{defn}}
\def\lm{\begin{lem}}
\def\elm{\end{lem}}
\def\zd{\begin{zad}}
\def\ezd{\end{zad}}
\def\prdl{\begin{predl}}
\def\eprdl{\end{predl}}
\def\crl{\begin{cor}}
\def\ecrl{\end{cor}}
\def\rm{\begin{rem}}
\def\erm{\end{rem}}
\def\fct{\begin{fact}}
\def\efct{\end{fact}}
\def\enm{\begin{enumerate}}
\def\eenm{\end{enumerate}}
\def\pmat{\begin{pmatrix}}
\def\epmat{\end{pmatrix}}

\frenchspacing
\righthyphenmin=2
%\usepackage{floatflt}
\captiondelim{. }





\begin{document}

\title{Конспект. Осень 2018}
\date{}
\author{}
\maketitle
\tableofcontents

\chapter{Полилинейная алгебра}
\section{Кватернионы}

Наша цель сейчас рассказать про геометрию трехмерного пространства используя при этом определённые алгебраические конструкции. А именно, ещё в XIX веке Уильям Роуэн Гамильтон стал искать аналогичную комплексным числам алгебраическую систему на трёхмерном пространстве.  

Рассмотрим подпространство в алгебре матриц $M_2(\mb C)$ вида
$$\mb H = \left\{\pmat \alpha & \beta \\ -\ovl{\beta} & \ovl{\alpha} \epmat \right\}.$$
Базис этого пространства, как вещественного векторного пространства, состоит из матриц 
$$ 1=\pmat 1 & 0 \\ 0& 1 \epmat, i= \pmat i & 0 \\ 0& -i \epmat, j=\pmat 0& 1 \\ -1 & 0 \epmat, k=\pmat 0 & i \\ i & 0\epmat. $$ 
Покажем, что это вещественная подалгебра в $M_2(\mb C)$ и следовательно ассоциативное кольцо. 
Для этого достаточно показать, что произведение базисных снова лежит в $\mb H$. Имеем $$i^2=j^2=k^2=-1 \text{ и } ij=k=-ji,$$ откуда $$ik= iij=-j=jii=-ki \text{ и } jk=-jji=i=-kj.$$ Таким образом $\mb H$ образует ассоциативную алгебру размерности 4 над $\mb R$.
 
\dfn[Алгебра кватернионов] $\mb H$ называется алгеброй кватернионов. 
\edfn
Мы больше не будем думать про кватернионы как про матрицы, а будем записывать их через $i,j,k$.

\zd Алгебра кватернионов не снабжается структурой $\mb C$-алгебры.
\ezd



Рассмотрим произведение двух чисто мнимых кватернионов $uv=-\lan u,v\ran+[u,v]$. Его вещественная часть совпадает с минус скалярным произведением векторов. Про мнимую часть мы поговорим отдельно.

\dfn[Векторное произведение] Пусть $v,u \in \mb R^3$ два вектора. Тогда их векторным произведением называется вектор $[v,u]$.
\edfn

Если расписать в координатах $u=x_1i+x_2j+x_3k$ и  $v=y_1i+y_2j+y_3k$, то векторное произведение задаётся формулой

$$[u,v]= (x_2y_3-x_3y_2)i + (x_3y_1-x_1y_3)j + (x_1y_2- x_2y_1)k= \begin{vmatrix} i& j&k \\ x_1 & x_2 & x_3 \\ y_1 & y_2 & y_3 \end{vmatrix} $$

\rm Операция $(v,u) \to [v,u]$ является билинейной и антисимметричной, то есть $[u,u]=0$ и, следовательно, $[u,v]=-[v,u]$.
\erm


\dfn[Сопряжённый кватернион] Пусть $x= a+bi+cj+dk$ кватернион. Определим вещественную или скалярную часть $\Re x=a$ и мнимую или векторную часть $v=\Im x= bi+cj+dk$ кватерниона. Сопряжённым кватернионом называется $\ovl{x}= a-bi-cj-dk= \Re x - \Im x =a-v$. 
\edfn

\dfn[Норма кватерниона] Определим норму кватерниона как $$||x||=\sqrt{x\ovl{x}}=\sqrt{ a^2+b^2+c^2+d^2}=\sqrt{\ovl{x}x}.$$
\edfn 


Норма кватерниона, как и модуль комплексного числа всегда положительны для ненулевых элементов. Это позволяет заметить, что

\dfn[Обратный кватернион] Если $0\neq x \in \mb H$, то $x^{-1}=\frac{\ovl{x}}{||x||^2}$. 
\edfn

Таким образом мы получили первый (и для нас единственный) пример некоммутативного кольца с делением. Такие кольца называются телами. Напоминаю, что алгебра для нас ассоциативна и с единицей. Неассоциативные алгебры представляют интерес. Например, можно взять $\mb R^3$, где в качестве умножения взято векторное произведение. Это пример неассоциативной алгебры или, точнее, алгебры Ли. Мы не будем обсуждать неассоциатные алгебры в связи с тем, что им находится применение либо внутри физических дисциплин, либо внутри самой математики. 

Какие ещё свойства есть у отображения нормы? Если следовать параллели с комплексными числами, то стоит посмотреть, что происходит с нормой произведения. Для того, чтобы не обременяться вычислениями сделаем небольшой трюк и на секунду вспомним матричное представление кватернионов. Заметим, что на матричном языке, норма -- это $||x||=\sqrt{\det x}$, откуда получаем

\lm[Норма мультипликативна] $||xy||=||x||||y||$. В частности, $||x^{-1}||=||x||^{-1}$.
\elm

\crl[Сумма четырёх квадратов] В любом коммутативном кольце произведение $(a^2+b^2+c^2+d^2)(e^2+f^2+g^2+h^2)$ снова есть сумма четырёх квадратов.
\ecrl

Теперь легко доказать 
\lm Отображение $x \to \ovl{x}$ является антиизоморфизмом алгебр, то есть $\ovl{ab}=\ovl{b}\ovl{a}$.
\proof Линейной ясна. Пусть $x,y \neq 0$. Тогда $$\frac{\ovl{y}\,\ovl{x}}{||y||^2||x||^2}=y^{-1}x^{-1}=(xy)^{-1}=\frac{\ovl{xy}}{||xy||^2}.$$
\elm

На самом деле и здесь можно было воспользоваться матричным представлением. А именно, можно заметить, что операция сопряжения совпадает на этом языке с транспонированием и сопряжением соответствующей комплексной матрицы. Вернёмся теперь к векторному произведению.


\lm[Свойства векторного произведения] Верны следующие свойства\\
1) Для любых $u,v \in \mb R^3$ верно $u\bot [u,v]$. Точнее $$u[u,v]= -||u||^2v+ \lan u,v\ran u$$
2) $|| [u,v]||= ||u||||v|||\sin \ffi |$, где $\ffi$ --  это угол между $u$ и $v$.
\elm
\proof Для того, чтобы посчитать скалярное произведение $\lan u, [u,v]\ran$ необходимо посчитать скалярную часть $u[u,v]$. 
$$u[u,v]= u (uv+ \lan u,v \ran)= u^2 v+ \lan u,v \ran u= -||u||^2 v+ \lan u,v \ran u$$
Последнее выражение, очевидно, чисто векторное.
Теперь 
$$||[u,v]||^2= -[u,v][u,v]= (uv + \lan u,v\ran)(vu + \lan u,v\ran)= ||u||^2||v||^2+ \lan u,v\ran^2 + \lan u,v\ran (uv+vu)=||u||^2||v||^2 - \lan u,v\ran^2= ||u||^2||v||^2(1-\cos^2 \ffi)$$
\endproof



\thrm Отображение $\mb H_{1}\to \GL_3(\mb R)$ заданное по правилу $x\to (y \to xyx^{-1})$ задаёт сюръективный  гомоморфизм из группы кватернионов единичной нормы в $\SO_3(\mb R)$. Ядро этого гомоморфизма состоит из $\{\pm 1\}$. Точнее, если единичный кватернион $x$  представим в виде $x=a+bv$, то соответствующее вращение есть вращение относительно  оси $\lan v \ran$ на угол $2\ffi$, где $\cos \ffi= a$, $\sin \ffi= b$ или тождественное преобразование вслучае $v=\pm 1$.
\ethrm
\proof Рассмотрим преобразование $L_x \colon \mb H \to \mb H$ вида $y \to xyx^{-1}$ Прежде всего покажем, что мы получили ортогональное преобразование $\mb R^4$. Имеем
 $$||xvx^{-1}||=||v||.$$
Теперь заметим, что преобразование $L_x$ сохраняет на месте вектор 1 и, следовательно, его ортогональное дополнение, то есть $\mb R^3$. Таким образом $L_x$ ограничивается на $\mb R^3$. Далее, очевидно, $L_xL_y= L_{xy}$. Осталось посчитать ядро гомоморфизма и явный вид отображения $L_x$. Заметим, что если $x=a+bv$, то $L_x$ оставляет $v$ на месте. Действительно, при $b\neq 0$ 
$$xbvx^{-1}=x(x-a)x^{-1}= x-a=bv.$$
Вычислим угол поворота. Для этого рассмотрим нормированный вектор  $u\bot v$ и $[u,v]$, которые образуют ортонормированный базис дополнения и посчитаем $xux^{-1}$ и $x[u,v]x^{-1}$. 
$$xux^{-1}=(a+bv)u(a-bv)= (a+bv)(au-[u,bv])=a^2u -ab[u,v]+ab[v,u]- b^2[v,[v,u]]=(a^2-b^2)u-2ab[u,v]$$
$$x[u,v]x^{-1}=(a+bv)[u,v](a-bv)= (a[u,v]+bu)(a-bv)=a^2[u,v]+abu-ab[u,v]v-b^2uv=(a^2-b^2)[u,v]+2abu $$
\endproof


\zd
Покажите, что отображение $(x,y) \to (z \to xzy^{-1})$ задаёт сюръективный гомоморфизм из декартового квадрата группы единичных кватернионов в группу $\SO_4(\mb R)$ с ядром $\{(1,1),(-1,-1)\}$.
\ezd

Обсудим теперь некоторое применение кватернионов.



\section{Задачи на максимизацию}

Теперь обратимся к вопросам, связанным с вещественными самосопряжёнными операторами. Для этого заметим, что с каждым самосопряжённым оператором $L$ на евклидовом пространстве можно связать билинейную симметричную  форму $\lan x,Ly\ran$ или, что эквивалентно, квадратичную форму $\lan x,Lx\ran$. Безусловно по квадратичной форме можно обратно восстановить оператор. 

Рассмотрим один из вопросов, связанных с такой конструкцией, а именно, рассмотрим задачу о нахождении нормы линейного оператора $L \colon U \to V$ между двумя евклидовыми пространствами. Для того, чтобы найти  норму необходимо найти $$\max_{x\neq 0}\sqrt{\frac{\lan Lx,Lx\ran}{\|x\|^2}}=\sqrt{\max_{x\neq 0}\frac{\lan L^*Lx,x\ran}{\|x\|^2}}=\sqrt{\max_{\|x\|=1} \frac{\lan L^*Lx,x\ran}{\|x\|^2}}.$$
Таким образом, нахождение нормы оператора свелось к задаче максимизации квадратичной формы на единичной сфере. Заметим, что максимум действительно достигается благодаря компактности сферы.

Оказывается, что довольно легко найти максимум или минимум квадратичной формы на сфере.



\thrm Пусть $V$ -- евклидово пространство, $A$ -- самосопряжённый оператор на $V$, а $q(x)=\lan x,Ax\ran$ -- соответствующая квадратичная форма. Тогда 
$$\max_{ x\in V } \frac{q(x)}{||x||^2}=\max_{\substack{ x\in V \\ ||x||=1}} q(x)=\lambda_1,$$
 где $\lambda_1$ - наибольшее собственное число оператора $A$ и достигается на собственном векторе $v_1$, соответствующему $\lambda_1$. Аналогично минимум равен минимальному собственному числу $A$. 
\proof
Пусть $v=\sum c_i e_i$, причём $1=||v||^2=\sum c_i^2$. Тогда $\lan Av,v\ran = \sum c^2_i \lambda_i $, что меньше $\lan A e_1,e_1\ran= \lambda_1= \sum \lambda_1 c_i^2$.
\endproof
\ethrm

Эта теорема, кроме, собственно, решения задачи, даёт геометрическую характеризацию первого собственного числа. Вопрос: можно ли аналогично охарактеризовать другие собственные числа? Ответ получается не таким простым, но, тем не менее, полезным.


\end{document}